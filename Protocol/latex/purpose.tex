\addcontentsline{toc}{section}{Purpose of this document}
\section*{Purpose of this document}

This document describes the binary protocols used by the ARSDK 3, and specifically by the \ARCode{ARNetworkAL}, \ARCode{ARNetwork}, \ARCode{ARDiscovery}, \ARCode{ARStream} and \ARCode{ARCommands} libraries.


This document is made for people who want to implement an ARSDK-compatible framework, without reading all the ARSDK source code and trying to figure out ``how it works''

\addcontentsline{toc}{section}{Conventions}
\section*{Conventions}

In this document, products are referred to by their network connection type, rather by their name, so the Bebop Drone and Jumping Sumo (and variants) are ``Wifi products'', while the Rolling Spider (and variants) is a ``BLE product''. If something is specific to a given product, its name will be written directly.

The connection is done between a controller (the computer, phone, tablet ...) and a device (the Bebop Drone, Rolling Spider ...). In the libraries, some objects are called ``c2d'' or ``d2c'', these notations means ``controller to device'' and ``device to controller'', respectively.

Inside the ARSDK, the products are sometimes called with other names than the actual product name. Here is a simple list of ARSDK alternative names for real names, which should help when reading the ARSDK source code:

\begin{table}[h]
\centering
\begin{tabular}{|c|c|}
  \hline
  Real Name & Alternatives \\
  \hline
  \hline
  Bebop Drone & ARDrone, ARDrone 3 \\
  \hline
  Jumping Sumo & JS \\
  \hline
  Rolling Spider & MiniDrone, RS \\
  \hline
  SkyController & SC \\
  \hline
  Airborne Night & MiniDroneEvoLight \\
  \hline
  Airborne Cargo & MiniDroneEvoBrick \\
  \hline
  Hydrofoil & MiniDroneEvoHydrofoil \\
  \hline
  Jumping Night & JSEvoLight \\
  \hline
  Jumping Race & JSEvoRace \\
  \hline
\end{tabular}
\caption{Name correspondence between product names and ARSDK internal names}
\end{table}

\addcontentsline{toc}{subsection}{Data endianness}
\subsection*{Data endianness}

For the \ARCode{ARNetwork}, \ARCode{ARNetworkAL}, \ARCode{ARStream} and \ARCode{ARCommands} library, all the data are sent on network in LITTLE\_ENDIAN byte order. This is done because most of the devices actually are in LITTLE\_ENDIAN mode, and thus avoids a lot of byteswapping.

The \ARCode{ARSAL} library provides conversion macros (similar to the \ARCode{htonl()}-family) in the\\
\ARFile{libARSAL}{Includes/libARSAL/ARSAL_Endianness.h} file.

%%% Local Variables:
%%% mode: latex
%%% TeX-master: "main"
%%% End:

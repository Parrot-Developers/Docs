\section{ARCommands protocol}

The \ARCode{ARCommand} library is a simple codec for sending binary data on the network.

\subsection{Command identifier}

The command is identified by its first 4 bytes:
\begin{itemize}
\item{Project or Feature ID (1 byte)}
\item{Class ID in the project/feature (1 byte)}
\item{Command ID in the class (2 bytes)}
\end{itemize}

In this document, commands are noted in the \ARCode{Project.Class.Command} notation. For example, the \ARCode{PCMD} command, in class \ARCode{Piloting} for project \ARCode{ARDrone3} is \ARCode{ARDrone3.Piloting.PCMD}.

\subsection{Command arguments}

The command arguments are directly packed after the command ID. Multi-bytes values are sent in little endian to avoid bytes swap in the product.

Here is a list of the supported argument types:
\begin{table}[h]
\centering
\begin{tabular}{|c|c|c|}
  \hline
  Type & Size (bytes) & Description \\
  \hline
  \hline
  u8 & 1 & unsigned 8bit value \\
  i8 & 1 & signed 8bit value \\
  \hline
  u16 & 2 & unsigned 16bit value \\
  i16 & 2 & signed 16bit value \\
  \hline
  u32 & 4 & unsigned 32bit value \\
  i32 & 4 & signed 32bit value \\
  \hline
  u64 & 8 & unsigned 64bit value \\
  i64 & 8 & signed 64bit value \\
  \hline
  float & 4 & IEEE-754 single precision \\
  double & 8 & IEEE-754 double precision \\
  \hline
  string & * & Null terminated string (C-String) \\
       & & (Variable size)\\
  \hline
  enum & 4 & Per command defined enum \\
       & & Coded as i32 on network \\
  \hline
\end{tabular}
\caption{Supported types for \ARCode{ARCommand} arguments}
\end{table}

\subsection{Commands definition}

Commands are defined in xml files. All the source code of the \ARCode{ARCommands} library is generated from these files.

The current xml files can be found in the \href{https://github.com/Parrot-Developers/libARCommands/tree/master/Xml}{\ARCode{<SDK>/libARCommands/Xml}} directory.

In this directory, each \ARCode{.xml} file correspond to a project or a feature. Each product only understand a given list of features. Here is a list of features implemented per product:

\begin{table}[h]
\centering
\begin{tabular}{|c|c|}
  \hline
  Product & Features \\
  \hline
  \hline
  Bebop Drone & ARDrone3, Common \\
  \hline
  Jumping Sumo & JumpingSumo, Common \\
  \hline
  Rolling Spider & MiniDrone, Common \\
  \hline
  SkyController & SkyController \\
  \hline
  Airborne Night & MiniDrone, Common \\
  \hline
  Airborne Cargo & MiniDrone, Common \\
  \hline
  Hydrofoil & MiniDrone, Common \\
  \hline
  Jumping Night & JumpingSumo, Common \\
  \hline
  Jumping Race & JumpingSumo, Common \\
  \hline
\end{tabular}
\caption{Features implemented by each product}
\end{table}

Note that implementing a features means that at least a subset of the feature is useful for the product, not that all commands in the feature are actually implemented!

Also note that the \ARCode{xxx_debug.xml} files contain debug commands that should be avoided. These commands can change from one version to another without notice, and can have unwanted behavior.

\subsubsection{The SkyController case}

The SkyController implements its own set of commands, even for common ones, as it can be connected to another device. When a SkyController is connected to a Bebop Drone, it will forward all \ARCode{ARDrone3.X.Y} and \ARCode{Common.Z.W} commands to the Bebop Drone, and will forward to the controller all the data coming from the Bebop (including the \ARCode{ARStream} data.

\subsection{Command attributes}

While not tied to the \ARCode{ARCommand} codec, certain commands can have other xml attributes, namely \ARCode{buffer}, \ARCode{timeout} and \ARCode{listtype}. These attributes are given as hints for an implementer on how the command is intended to be used.

\subsubsection{buffer}

The value of this attribute can be either \ARCode{NON_ACK}, \ARCode{ACK} or \ARCode{HIGH_PRIO}, defaulting to \ARCode{ACK} if not given. It gives a hint about the destination buffer for the command.

For the Bebop Drone, the \ARCode{NON_ACK} buffers are 10 (c2d) and 127 (d2c), the \ARCode{ACK} buffers are 11 (c2d) and 126 (d2c), and the \ARCode{HIGH_PRIO} buffer is the 12 (c2d).

This is only a hint, and the product will decode any \ARCode{ARCommand} on any \ARCode{ARNetwork} buffer, as long as the buffer is not used for \ARCode{ARStream}.

\subsubsection{timeout}

The value of this attribute can be either \ARCode{POP}, \ARCode{RETRY} or \ARCode{FLUSH}, defaulting to \ARCode{POP} if not given.

For acknowledged data, if a timeout happens (after the retries from \ARCode{ARNetwork}), there is three possible answers:
\begin{itemize}
\item{POP: Pop the data from the fifo, and continue with the next data (default).}
\item{RETRY: Retry the data. This leads to infinite retries of the current data.}
\item{FLUSH: Flush the entire \ARCode{ARNetwork} buffer. This can be useful if the next data depends on the current one.}
\end{itemize}

This information has no effect on non acknowledged data.

\subsubsection{listtype}

The value of this attribute can be either \ARCode{NONE}, \ARCode{LIST} or \ARCode{MAP}, defaulting to \ARCode{NONE} if not given.

\ARCode{LIST} commands are sent multiple times, and a list of all the received value should be created. The \ARCode{ARController} library uses a hash map with a locally generated integer key to emulate this.

\ARCode{MAP} commands are sent multiple times, and their first argument should be used as a key in a map of received values.

In both cases, clearing the map/list before requesting the data is the responsibility of the receiver, unless stated otherwise in a specific command implementation.

\subsection{Commands list}

Listing all the commands in this document would be too long. The format used in the \ARCode{.xml} files is designed to be human readable, with inline comments about every commands and arguments.

Some of the important commands are listed below.

\subsubsection{\ARCode{Common.Settings.AllSettings} \&\\\ARCode{Common.Common.AllStates}}

Normally, the product will send state and settings updates on the go. To synchronize them, you will have to request a full snapshot of the settings and the states of the product during the initialization.

When receiving these commands, the product will send all its settings (or states), then it will send a final command, saying that all the settings or state were sent.

The SkyController uses the \ARCode{SkyController.Settings.AllSettings} and \ARCode{SkyController.Common.AllStates} commands instead.

\begin{table}[h]
\centering
\resizebox{\textwidth}{!}{%
\begin{tabular}{|c|c|}
  \hline
  Request & Final answer \\
  \hline
  \hline
  \ARCode{Common.Settings.AllSettings} & \ARCode{Common.SettingsState.AllSettingsChanged} \\
  \hline
  \ARCode{Common.Common.AllStates} & \ARCode{Common.CommonState.AllStatesChanged} \\
  \hline
  \hline
  \ARCode{SkyController.Settings.AllSettings} & \ARCode{SkyController.SettingsState.AllSettingsChanged} \\
  \hline
  \ARCode{SkyController.Common.AllStates} & \ARCode{SkyController.CommonState.AllStatesChanged} \\
  \hline
\end{tabular}
}
\caption{Final command for each Settings/State request}
\end{table}

\subsubsection{\ARCode{Common.Common.CurrentDate} \&\\\ARCode{Common.Common.CurrentTime}}

These commands set the date/time on the product, which in turn is set into the media and PUD (for \ARCode{ARDrone Academy} files).

The argument to these commands is an \href{https://en.wikipedia.org/wiki/ISO_8601}{ISO-8601} formatted string, with the following formatters:
\begin{itemize}
\item{\ARCode{"yyyy-MM-dd"} for \ARCode{Common.Common.CurrentDate}. Ex: \ARCode{2015-08-27}.}
\item{\ARCode{"'T'HHmmssZZZ"} for \ARCode{Common.Common.CurrentTime}. Ex: \ARCode{T101527+0200}.}
\end{itemize}

For compatibility purposes, you should always sent both commands to the product, not juste the \ARCode{Common.Common.CurrentDate} one. The order is irrelevant on newer firmwares, but older ones need the \ARCode{Common.Common.CurrentTime} to be sent after.

Note that you should generate both strings from a single timestamp to avoid any loop error at midnight.

\subsubsection{\ARCode{ARDrone3.MediaStreaming.VideoEnable} \&\\\ARCode{JumpingSumo.MediaStreaming.VideoEnable}}

These commands will enable or disable the video streaming from the product. When connecting, the streaming is disabled (which lets a high bandwidth to transfer all the data needed by the \ARCode{Common.Settings.AllSettings} and \ARCode{Common.Common.AllStates} commands). You should enable the video only when needed (when you actually display it !).

In FreeFlight, the video is also disabled in the following cases:
\begin{itemize}
\item{When the user browses the internal memory of the drone.}
\item{When the user downloads media from the drone.}
\item{When the user sends an update file to the drone.}
\end{itemize}


\subsection{Using the ARCommandsParser to generate your own code}

The code generator is split into a parser, and a generator part. The parser is written in Python, and can be found in the \ARCode{ARSDKBuildUtils} repository:
\ARFile{ARSDKBuildUtils}{Utils/Python/ARCommandsParser.py}

It defines a \ARCode{parseAllProjects} function, which takes 4 arguments:
\begin{itemize}
\item{projects: A list of strings, list of the projects to parse. If this list contains the ``all'' string, then all projects are parsed, regardless of the content of the other elements.}
\item{pathToARCommands: The path to the \ARCode{ARCommands} library root (not the \ARCode{Xml} directory !)}
\item{genDebug: if True, then the \ARCode{xxx_debug.xml} files are also parsed. Defaults to False. Should not be used}
\item{mergeDebugProjectsInReleaseProjetcs: if True, then the debug commands will be merged with the release commands in single projects instead of being in separated debug projets. Defaults to False. Should not be used}
\end{itemize}

And returns a list of \ARCode{ARProject} objects. This class (and all other internally used class) are fairly straightforward to understand, and thus are not described in depth here.

A simple example of iterating on all commands can be seen at lines 519-554 of \ARFile{libARCommands}{Xml/generateLibARCommands.py}


%%% Local Variables:
%%% mode: latex
%%% TeX-master: "main"
%%% End:
